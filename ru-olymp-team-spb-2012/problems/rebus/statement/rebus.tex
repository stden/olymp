\begin{problem}{Ребус}{rebus.in}{rebus.out}{2 секунды}{256 мегабайт}

%Автор задачи: Антон Банных
%Автор условия: Борис Минаев

На прошлом уроке английского Пете задали домашнее задание. Оно заключалось
в разгадывании ребусов. Каждый ребус~--- это последовательность картинок. 
С двух сторон от каждой картинки могут располагаться апострофы. Каждая картинка
обозначает некоторое слово. Предположим, что перед некоторой картинкой 
нарисовано $i$ апострофов, а после нее $j$ апострофов. Это значит, что у слова, которое
сопоставляется картинке, необходимо убрать $i$ букв с начала и $j$ с конца,
а оставшуюся его часть записать вместо картинки и апострофов. Так необходимо
сделать с каждой картинкой и окружающими ее апострофами. После этого нужно
<<склеить>> получившиеся кусочки в одно слово. Оно и будет разгадкой ребуса.

У Пети нет проблем с тем, чтобы сопоставить каждой картинке слово. Но ему очень
лень заниматься отбрасыванием лишних букв и склеиванием слов. Поэтому он
попросил вас помочь ему. Дана строка, которая состоит из маленьких 
латинских букв, апострофов (символов с кодом 39), а также пробелов,
которые разделяют слова. Апостроф относиться к некоторому слову, если между 
ними нет пробела. Если апостроф стоит слева от слова, то у него необходимо
убрать одну букву с начала, если справа~--- с конца. Потом необходимо склеить
все слова в одно.
 
Например, пусть дана строчка <<team '\,'\,'\,'\,school 
'\,'\,olympiad'\,'\,'>>. В первом слове ничего изменять не надо, так как
к нему не относится ни один апостроф. Во втором необходимо убрать первые четыре
буквы и получить <<ol>>, из третьего слова получится <<ymp>>. После скеливания
трех кусочков получим строку <<teamolymp>>.

\InputFile
В первой строке входного файла дан ребус, длиной не более $100$ символов, 
который необходимо решить. Гарантируется, что в строчке присутствуют только 
апострофы (код символа~--- 39), пробелы и маленькие латинские буквы,
а также, что ребус корректен~--- нет слова, у которого нобходимо удалить
букв больше, чем его длина. 

\OutputFile
Выведите одно слово~--- ответ на ребус.

\Examples

\begin{example}%
\exmp{
team '\,'\,'\,'\,school '\,'\,olympiad'\,'\,'
}{
teamolymp
}%
\end{example}

\end{problem}
