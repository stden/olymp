\begin{problem}{Рекламное объявление}{advert.in}{advert.out}{2 секунды}{256 мегабайт}

%Автор задачи: Георгий Корнеев
%Автор условия: Николай Ведерников

Ивану с детства нравились газеты. У него даже была мечта стать главным редактором газеты. 
Однажды ему представился шанс осуществить свою мечту. 
Чтобы устроиться на работу в издательство, ему необходимо выполнить тестовое задание~--- сверстать рекламное объявление. 

Задано поле шириной $W$ и высотой $H$.
Объявление должно состоять из одной или нескольких строк,
в которых необходимо разместить в заданном порядке $N$ слов. 
Про $i$-е слово известно, что при печати в стандартном масштабе оно занимает прямоугольник
шириной $a_i$ и высотой $b_i$.

Чтобы объявление выглядело красиво, все слова в нем должны быть напечатаны в одном масштабе. 
При печати в масштабе 
$k$ размеры всех слов умножаются на $k$. 
Если исходно слово занимало прямоугольник $a_i \times b_i$, то при печати в масштабе
$k$ оно занимает прямоугольник размером $(k \cdot a_i)\times(k \cdot b_i)$.
Кроме того, если в строке более одного слова, то все слова в ней должны иметь 
одинаковую высоту. Разумеется, ни одно слово не должно выходить за границы поля.

На рисунке приведен пример красивого объявления с тремя словами.

\begin{center}
	\includegraphics{../problems/advert/statement/pics/pic.1}
\end{center}

Помогите Ивану найти максимальный масштаб, при котором можно сверстать объявление, 
которое удовлетворяет этим критериям. Обратите внимание, что
менять порядок слов нельзя, они должны читаться по строкам
сверху вниз, слева направо в том порядке, в котором заданы.

\InputFile
В первой строке входного файла дано три числа: $N$, $W$ и $H$ ($1 \le N \le 100\,000$, $1 \le W, H \le 10^9$)~---
число слов в объявлении, длина и высота объявления.           
В следующих $N$ строках дано по два целых числа, в $i$-й из
них заданы $a_i$ и $b_i$ ($1 \le a_i, b_i \le 10^9$)~--- ширина и высота $i$-го слова.          

\OutputFile
Выведите одно вещественное число $k$~--- максимальный масштаб. Ответ требуется вывести с абсолютной или относительной
погрешностью не более $10^{-9}$. Это значит, что если правильный ответ $a$, а вы вывели $p$, то
ваш ответ будет засчитан как правильный, если $\frac{|a-p|}{\max(|a|, 1)} \le 10^{-6}$.

\Examples

\begin{example}%
\exmp{
3 10 7
4 3
3 2
4 2
}{
1.4
}%
\exmp{
2 10 1
2 1
3 2
}{
0.333333333
}%       
\end{example}


\end{problem}
