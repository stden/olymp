\begin{problem}{Лягушонок Билли}{frog.in}{frog.out}{2 секунды}{256 мегабайт}

%Автор задачи: Анна Малова
%Автор условия: Демид Кучеренко

Лягушонок Билли сидел на камне и любовался на закат, когда понял, что проголодался. Он огляделся и
с удивлением обнаружил, что в ручье около него копошатся мошки. Ручей представляет собой прямую,
на которой расположен и камень, на котором сидит Билли.
Лягушонок был очень голоден, и потому захотел съесть всех мошек. У Билли очень длинный язык, поэтому
он может, не спрыгивая с камня, съесть любую мошку (но только одну за раз). 

Однако высовывать язык на большие расстояния не так-то просто, лягушонок на каждый сантиметр высунутого языка 
тратит одну единицу энергии. Каждый раз, когда Билли съедает мошку из какой-то точки происходит следующее:
все мошки, сидящие слева от съеденной мошки, и все мошки, сидящие справа от нее
в ужасе отпрыгивают от места событий на один сантиметр вдоль ручья.
Мошки, которые сидят в той же точке, что и съеденная, настолько шокированы этим событием, что не двигаются. 

\begin{center}
	\includegraphics{pics/pic.0}
\end{center}

Если мошка в какой-то момент времени прыгает на камень, где сидит Билли, то Билли тут же съедает ее не тратя энергии.
При этом другие мошки не перемещаются.

Лягушонок Билли хочет понять~--- какое минимальное количество единиц энергии ему потребуется для того, 
чтобы съесть всех мошек. Помогите ему это выяснить.

\InputFile
В первой строке входного файла задано одно натуральное числа $n$ ($1 \le n \le 100{\,}000$)~--- количество мошек.
Во второй строке входного файла задано $n$ натуральных чисел~--- расстояния каждой из мошек до камня. Известно,
что все мошки находятся на одной прямой по одну сторону от камня. Расстояния даны в порядке неубывания.
Расстояния не превышают $10^9$.

\OutputFile
Выведите одно число~--- минимальное количество единиц энергии, которое потребуется Билли, 
чтобы съесть всех мошек.

\Examples
\begin{example}%
\exmp{
4
2 2 4 4 
}{
8
}%
\end{example}

\Note
Пояснение к примеру. Сначала Билли съест одну мошку, сидящую в точке 4. 
Другая мошка, сидящая в этой точке не сдвинется, обе мошки из точки 2 сдвинутся 
в точку 1. После того, как он съест вторую мошку в точке 4, обе мошки из 
точки 1 отпрыгнут в точку 0, где и будут сразу съедены.

\end{problem}
