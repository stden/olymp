\begin{problem}{День рождения викинга}{vikings.in}{vikings.out}{2 секунды}{256 мегабайт}

%Автор задачи: Демид Кучеренко
%Автор условия: Антон Ковшаров

Немногие знают, но помимо боевого и трудового топора викинги используют кухонный топор. Сегодня у Хильдсвегсамара Харфагра по прозвищу 
Кровавый Топор день рождения. Хильдсвегсамар, как всегда, зашел к своей бабушке Одаудлегье подкрепиться и обнаружил, что бабушка испекла
огромный круглый торт и положила его на стол. Теперь его необходимо разделить на две  части для Хильдсвегсамара и его бабушки. 

В семье Харфагров принято делить торты одним ударом топора. После удара топора образуется разрез в том месте, где ударил викинг.
Разрез представляет собой отрезок, длина которого не больше длины лезвия топора. 
Торт оказывается поделен на две части, если разрез соединяет две точки на границе торта. Так как
длина топора фиксирована, возможно, торт не удастся разделить на две равные части за один разрез. 
Поэтому викинги хотят разделить одним разрезом торт так, чтобы им достались по возможности наиболее 
близкие по площади части. 

Хильдсвегсамар быстро догадался, что торт нужно разделить по хорде максимальной длины, не превосходящей длины лезвия, но не может найти где 
именно нужно резать. Введем на столе прямоугольную декартову систему координат с центром, совпадающим с центром торта. 
Помогите Хильдсвегсамару найти две точки на границе торта, через которые должен проходить разрез, чтобы  разделить торт наиболее честно.

\InputFile
В первой строке входного файла даны два вещественных числа $R$ и $L$~--- радиус торта и длина лезвия соответственно с не более чем тремя
знаками после десятичной точки ($1 \le R, L \le 1000$).

\OutputFile
Первая строка выходного файла должна содержать координаты первой точки на границе, вторая строка координаты второй точки. Расстояние от 
точек до центра должно отличаться от $R$ не более чем на $10^{-6}$. Расстояние между точками должно отличаться от максимально возможной
хорды не превосходящей $L$ не более чем на $10^{-6}$.

\Examples

\begin{example}%
\exmp{
1.0 2.0
}{
0 -1.0
0 1.0
}%
\exmp{
1.0 1.71
}{
-1.0 0
0.46205 0.886853
}%
\end{example}

\begin{center}
	\includegraphics{pics/pic.1}\\
	Одно из возможных расположений разреза во втором примере.
\end{center}



\end{problem}
