\begin{problem}{Племя тив}{tiv.in}{tiv.out}{2 секунды}{256 мегабайт}

%Автор задачи: Михаил Дворкин
%Автор условия: Демид Кучеренко

Каждый год профессор Иванов ездит в Африку с целью изучить племена, которые там проживают. В этом 
году он ездил в гости к племени тив. Профессор довольно быстро научился понимать их язык,
выучил многие их обряды, однако, он никак не мог понять записанные цифрами тив числа. 
Как и мы, члены племени используют позиционную систему счисления с основанием 10.
Но цифры в племени тив обозначают символами, не похожими на обычные цифры от 0 до 9.

Профессор обозначил эти символы буквами от `\texttt{a}' до `\texttt{j}',
но не может понять, какой цифре соответствует какой символ. Тогда вождь племени дал
ему список из $n$ неотрицательных чисел, записанных без ведущих нулей, и сказал, что 
числа в нем отсортированы строго по возрастанию.

Помогите профессору восстановить по этому списку какое-нибудь соответствие символов цифрам.

\InputFile
В первой строке входного файла дано одно натуральное числа $n$ ($2 \leq n \leq 10$)~--- количество слов в списке.
Следующие $n$ строк содержат выданные вождем числа племени тив, по одному числу в строке. Длина каждого 
числа не превышает 9.

\OutputFile
В первой строке файла выведите <<\texttt{Yes}>>, если ответ существует, в этом случае в следующей строке 
выведите цифры, которые соответствуют символам, обозначенным `\texttt{a}'..`\texttt{j}', в этом порядке. 
Если существует несколько ответов, то выведете любой из них. 

Если профессор понял что-то неправильно, и ответа не существует, выведете <<\texttt{No}>>.

\Examples                                                                                                  
\begin{example}%
\exmp{
4
a
da
dd
cc
}{
Yes
1 0 3 2 4 5 6 7 8 9
}%
\exmp{
4
a
j
jb
ac
}{
No
}%
\end{example}

\end{problem}
