\begin{problem}{Реклама на заборе}{checkpaint.in}{checkpaint.out}{2 секунды}{256 мегабайт}

%Автор задачи: 
%Автор условия: Андрей Комаров

Иван живет в небольшом симпатичном домике в деревне. Вдоль его участка расположен забор,
который недавно был выкрашен в красный цвет. Но тут в деревню к Ивану пришла цивилизация
в лице рекламного агента, расклеивающего всюду свои объявления. 
И его забор постигла та же участь. 

Каждый день на его забор приклеивают новое объявление.
Таким образом за последние $n$ дней на забор наклеено уже $n$ объявлений и Ивану кажется, 
что рекламой заклеен уже весь забор, состоящий из $m$
досок. Доски пронумерованы вдоль забора от 1 до $m$.

Оказалось, что в каждый из $n$ дней когда приходил рекламный агент и приклеивал
объявление, сосед Ивана Петр записывал, какие доски оказывались заклеены этим объявлением. 
А именно, выяснилось что в $i$-й день очередное объявление было наклеено 
таким образом, что занимало доски с $l_i$-й по $r_i$-ю включительно. 
При этом рекламный агент вполне мог заклеить новым объявлением полностью или
частично свое же собственное объявление.

Для составления жалобы в администрацию деревни Ивану необходимо удостовериться, что
рекламой заклеен весь забор. Помогите ему выяснить, так ли это.

\InputFile
В первой строке входного файла даны два натуральных числа $m$ и $n$ ~--- 
число досок в заборе и число дней, в течение которых вел свои наблюдения Петр
($1 \le m \le 10\,000$, $1 \le n \le 1000)$.
Далее, в $n$ строках заданы целые числа $l_i$, $r_i$ ($1 \le l_i \le r_i \le m$), $i$-я
пара чисел описывает отрезок забора, который заклеивались объявлением в $i$-й день.

\OutputFile
Выведите <<\texttt{YES}>>, если весь забор был заклеен объявлениями, и <<\texttt{NO}>>
в противном случае.

\Examples

\begin{example}%
\exmp{
3 3
1 1
2 3
3 3
}{
YES
}%
\end{example}

\end{problem}
