\begin{problem}{Академия Джедаев}{jediacademy.in}{jediacademy.out}{2 секунды}{256 мегабайт}

%Автор задачи: Федор Царев
%Автор условия: Антон Ковшаров

Чтобы стать джедаем, необходимо овладеть многими теоретическими и практическим навыками.
В Академии Джедаев можно обучиться всему необходимому, если, конечно, у тебя есть способности.

Новый студент академии Фил очень способный и обучается индивидуально. Он вправе самостоятельно 
составлять свое расписание. Фил очень любознателен, поэтому он все свое время посвящает учебе.
Фил не учится только в те моменты, когда переходит из одного корпуса академии в другой или
идет из общежития в один из корпусов.
 
Академия состоит из двух корпусов, в одном из которых обучают теоретическим навыкам, а в другом~---
практическим. Путь из одного корпуса в другой занимает ровно $a$ минут. Изучение любого 
навыка занимает ровно $b$ минут. В начальный момент Фил находится в общежитии, 
путь от которого до каждого из корпусов также занимает $a$ минут.

Разумеется, навыки нельзя изучать в произвольном порядке. Например, прежде чем овладеть 
световым мечом, нужно сначала изучить основы оптики и искусство рукопашного боя. Филу
важно учесть это при составлении своего расписания. 

Фил хочет как можно скорее стать джедаем, но для этого ему нужно обучиться всем необходимым навыкам.
Прежде чем приступить к учебе, он хочет узнать через какое минимальное число минут он сможет изучить 
все навыки. Помогите ему это выяснить. После окончания учебы Фил сразу же отправится
сражаться со злом, возвращаться в общежитие ему не требуется.

\InputFile                                                     
В первой строке входного файла задано целое число $n$~--- количество навыков, которым нужно обучиться в академии
($1 \le n \le 10^5$). Все навыки пронумерованы от 1 до $n$.

В следующих $n$ строках описаны требования для изучения каждого навыка. В начале $i$-й из
этих строк записано число 1 или 2, оно указывает в каком корпусе можно изучить $i$-й навык. 
Затем следует число $k$~--- количество навыков, необходимых для изучения навыка $i$. Затем 
в этой же строке следует $k$ целых чисел~--- навыки, 
необходимые для изучения навыка $i$. 
Гарантируется, что сумма чисел $k$ по всем навыкам не превосходит $10^5$. 

В следующей строке задано два целых числа $a$~--- время в минутах на путь из одного корпуса
в другой или из общежития до корпуса, и $b$~--- время на изучение одного навыка ($1 \le a, b \le 10^4$).

Гарантируется, что существует такой порядок изучения всех навыков, при котором каждый
раз при изучении очередного навыка все необходимые для него уже изучены. 

\OutputFile                        
В единственной строке выходного файла выведите минимальное время в минутах, которое потребуется
Филу, чтобы стать джедаем.

\Example
\begin{example}%
\exmp{
6
1 3 3 4 5
2 1 4
2 2 5 6
1 1 6
1 0
1 0
15 40
}{
285
}%
\end{example}

\end{problem}
