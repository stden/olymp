\begin{problem}{Швабра}{rag.in}{rag.out}{2 секунды}{256 мегабайт}

%Автор задачи: Елена Андреева 
%Автор условия: Анна Малова

Первокурсник Рома приехал в общежитие и, удивившись беспорядку 
в комнате и толстому слою пыли на полу, начал наводить порядок. 
Первым делом он решил вымыть пол. 
Для этого Рома в магазине приобрел инновационную швабру.

Изначально моющая часть швабры имела идеальную прямоугольную форму, но в процессе
перевозки из магазина в общежитие у нее отломился один из углов. 
Таким образом, теперь она представляет собой многоугольник,
граница которого состоит из двух соседних сторон прямоугольника,
фрагментов двух оставшихся сторон и ломаной,
соединяющей концы этих фрагментов.

\begin{center}
\includegraphics{pics/rag.1}
\end{center}

Рома живет в большой прямоугольной комнате.
Рома провел сломанной шваброй от одной стороны комнаты до другой, 
не отрывая ее от стены, так что в результате отломанный угол швабры
оказался в углу комнаты. При этом часть соответствующей прямоугольной полосы пола 
в углу осталась невымытой. 

\begin{center}
\includegraphics{pics/rag.2}
\end{center}

Рома считает, что все точки, в которых в какой-то момент находилась
моющая часть швабры оказались вымыты. Теперь он решил выяснить, какая часть этой полосы 
осталась грязной. 

\begin{center}
\includegraphics{pics/rag.3}
\end{center}

Помогите ему вычислить площадь этой части.
Можете считать, что размер комнаты, в которой живет Рома, существенно больше 
размеров моющей части швабры.

\InputFile
Первая строка входного файла содержит два целых числа $w$ и $h$~--- размеры моющей
части швабры до повреждения ($2 \le w, h \le 10^5$).

Вторая строка содержит целое число $n$~--- число вершин ломаной, 
соединяющей соседние стороны швабры ($2 \le n \le 10^5$).
В $i$-й из следующих $n$ строк заданы два целых числа 
$x_i$ и $y_i$ ($1 \le x_i < w$, $1 \le y_i < h$, за исключением $y_1 = h$, $x_n = w$)~--- координаты $i$-й
вершины ломаной.
Ломаная не имеет самопересечений и самокасаний.

Координаты введены таким образом, что стена, вдоль которой Рома провел
шваброй, соответствует прямой $y=h$.
 
\OutputFile
В выходной файл выведите площадь невымытой части пола
с абсолютной или относительной погрешностью не более $10^{-6}$.
Это значит, что если правильный ответ $a$, а вы вывели $p$, то
ваш ответ будет засчитан как правильный, если $\frac{|a-p|}{\max(|a|, 1)} \le 10^{-6}$.

\Examples

\begin{example}%
\exmp{
9 7
9
3 7
4 5
5 6
4 4
5 2
6 4
7 2
8 3
9 2
}{
18.0
}%
\end{example}

\end{problem}
