\begin{problem}{Экзамен}{exam.in}{exam.out}{2 секунды}{256 мегабайт}

%Автор задачи: Глеб Евстропов
%Автор условия: Михаил Пядеркин

Мария Петровна работает преподавателем в университете. 
В середине семестра она дала своим студентам контрольную работу, в которой было $n$
заданий, упорядоченных по возрастанию сложности. 

Мария Петровна знает, что каждый из ее студентов относится к одной из трех категорий:
\begin{itemize}
\item \textit{Трудолюбивый} студент решает на контрольной задачи в порядке увеличения сложности, начиная с самой легкой, 
и не пропуская ни одной задачи, пока не закончится время контрольной.
\item \textit{Умный} студент решает на контрольной задачи в порядке уменьшения сложности, начиная с самой сложной,
и не пропуская ни одной задачи, также пока не закончится время контрольной.
\item \textit{Плохой} студент списывает ровно одну задачу у какого-либо трудолюбивого или умного студента. 
При этом, два плохих студента не могут списать одну и 
ту же задачу у одного и того же человека.
\end{itemize}
Трудолюбивые и умные студенты всегда правильно решают все задачи, которые успевают.

Мария Петровна очень не любит студентов, которые списывают. Поэтому, если она узнает, что кто-то у кого-то списал,
то обоим студентам она эту задачу не засчитывает. 
По итогам контрольной Мария Петровна внесла в ведомость зачтенные задачи для каждого студента и
про каждую задачу посчитала, скольким студентам она ее зачла. 

Поскольку приближаются экзамены, Мария Петровна решила оценить число
плохих студентов в группе, чтобы спланировать время на пересдачи.
К сожалению, она потеряла ведомость с зачтенными задачами и сохранила 
лишь листок, в котором записано для каждой задачи, какому числу студентов
эта задача зачтена. 

Мария Петровна интересуется, какое минимальное
число плохих студентов может быть в группе.
Помогите ей это выяснить.

\InputFile
Входной файл содержит несколько тестов.
В первой строке входного файла содержится целое число $T$~--- число тестов.
Далее следуют описания тестов.

Описание каждого теста состоит из двух строк.
В первой строке описания задано целое число $n$~--- количество заданий в контрольной работе.
Далее для каждого задания, в порядке возрастания сложности, указано число~$a_i$~--- количество студентов, 
которым зачтена соответствующая задача ($0 \le a_i \le 10^9$). 

Гарантируется, что суммарное число 
заданий во всех тестах не превосходит $10^5$, а также то, что
в контрольной была хотя бы одна задача.

\OutputFile

Для каждого теста выведите единственное число~--- искомое минимальное 
возможное число плохих студентов.

\newpage

\Example
\begin{example}%
\exmp{
2
4
3 2 0 1
3
1 5 1
}{
0
3
}%
\end{example}

\Note
В первом тесте у Марии Петровны могло быть три трудолюбивых студента, два из которых решили по две задачи, а один --- только одну, и еще один
умный студент, который решил самую сложную задачу (но ничего больше решить не успел).

Во втором тесте у Марии Петровны мог быть один умный студент, который решил две последних задачи, 
четыре трудолюбивых студента, которые решили по две первые задачи и три плохих студента, которые списали первую задачу,
каждый у одного из трудолюбивых студентов. В результате из семи решений первой задачи шесть было аннулировано,
зачтенным это задание осталось только у трудолюбивого студента, у которого никто не списал.

\end{problem}
