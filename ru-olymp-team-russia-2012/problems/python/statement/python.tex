\begin{problem}{Длинный питон}{python.in}{python.out}{2 секунды}{256 мегабайт}

%Автор задачи: Денис Кириенко 
%Автор условия: Анна Малова

\epigraph{Раз, два, левой, правой,\\
дважды два~--- очень просто\\
измеряются удавы\\
пятью пять~--- любого роста}{Попугай}

После того, как Мартышка и Попугай досконально исследовали длину Удава, им 
стало очень скучно. Тут Слоненок вспомнил, что в лесу живет еще
Питон, которого тоже можно измерять! Друзья сразу же отправились на его поиски.

Питон, как и Удав, тоже целый, поэтому его нельзя мерить половинками.
Измерив Питона, Мартышка и Попугай узнали, что в Питоне помещается 
$n$ целых Попугаев или $m$ целых Мартышек. Обрадованная Мартышка убежала 
сообщить полученный результат Слоненку.
Когда она ушла, Попугая заинтересовал следующий вопрос:
а сколько раз он помещается в одной Мартышке?

Так как Мартышка убежала, и измерить ее он не может, Попугай решил попытаться выяснить, 
сколько целых Попугаев может поместиться в одной Мартышку, используя результаты
измерения Питона. По заданным $n$ и $m$ выясните, какое минимальное
и максимальное число целых Попугаев может помещаться в одной Мартышке.

\InputFile

Во входном файле заданы два целых числа $n$ и $m$, каждое на своей строке~--- количество
Попугаев и Мартышек в Питоне, соответственно ($1 \le n, m \le 10^9$).

\OutputFile
В выходной файл выведите два числа~--- минимальное и максимальное количество
целых Попугаев в одной Мартышке.

\Example

\begin{example}%
\exmp{
38
5
}{
6
7
}%
\end{example}

\end{problem}
