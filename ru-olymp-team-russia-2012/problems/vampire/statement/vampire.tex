\begin{problem}{Вампирские числа}{vampire.in}{vampire.out}{2 секунды}{256 мегабайт}

%Автор задачи: Никита Иоффе
%Автор условия: Никита Иоффе

Илья увлекается математикой. Недавно он прочитал про \emph{вампирские числа}. Они настолько восхитили Илью, что
теперь он постоянно придумывает задачи, связанные с этими числами, и пытается их решить.

Число $a$, десятичная запись которого состоит из $n$ цифр ($n$ четно), называется вампирским, если его можно представить в виде 
произведения двух $n/2$-значных чисел $b$ и $c$, причем используя все цифры $b$ и $c$ 
можно записать число $a$. Каждую цифру при этом разрешается использовать столько раз, сколько
раз она суммарно встречается в $b$ и в $c$.
Числа $b$ и $c$ называются \textit{клыками} числа $a$.

Например,  число $6880$~--- вампирское, так как $6880 = 80 \times 86$,
а число $1023$~--- нет.

Для его новой задачи Илья попросил вас найти $k$ различных вампирских чисел, состоящих из $n$ цифр.

\InputFile
В единственной строке входного даны два числа $k$ и $n$~--- требуемое количество вампирских чисел и 
количество цифр в каждом из них соответственно ($1 \le k \le 100$, $4 \le n \le 100$, $n$~--- четно).

\OutputFile
В выходной файл выведите $k$ различных $n$-значных вампирских числа в формате \texttt{$A_i$=$B_{i}$x$C_{i}$}, где $A_i$~---
$i$-е из найденных вампирских чисел, $B_i$ и $C_i$~--- его клыки 
(между $B_i$ и $C_i$ следует вывести маленькую латинскую букву <<\texttt{x}>>). 

Если ответов несколько, то разрешается вывести любой из них.
Гарантируется, что для приведенных во входном файле $n$ и $k$ существует $k$ различных
$n$-значных вампирских чисел.

\Examples

\begin{example}%
\exmp{
1 6
}{
125460=246x510
}%
\exmp{
3 4
}{
1260=60x21
1395=15x93
1530=30x51
}%
\end{example}

\end{problem}
