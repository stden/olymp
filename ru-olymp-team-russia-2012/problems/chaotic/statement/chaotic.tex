\begin{problem}{Хаотическая перестановка}{chaotic.in}{chaotic.out}{2 секунды}{256 мегабайт}

%Автор задачи: Игорь Пышкин
%Автор условия: Павел Кунявский

Сегодня Васю заставили убираться в классе. Устав наводить порядок, он решил,
что теперь он просто должен в качестве компенсации устроить где-нибудь хаос. 
И тут ему на глаза попалась 
написанная учителем на доске перестановка чисел от 1 до $n$.
Напомним, что перестановкой чисел от 1 до $n$ называется последовательность из $n$ 
чисел, в которой каждое из них встречается ровно один раз.

Вася считает, что три подряд идущих элемента находятся в порядке, если они
упорядочены либо по возрастанию, либо по убыванию. Он называет перестановку
хаотической, если никакая тройка подряд идущих элементов не находится в порядке.

Вася решил изменить перестановку на доске, сделав ее хаотической.
Для этого он решил не более $n$ раз поменять местами два соседних элемента
в перестановке.

Помогите Васе сделать перестановку хаотической, пока не пришел учитель и не наказал его
за то, что он занимается ерундой вместо уборки.

\InputFile

Во входном файле задана исходная перестановка, которая написана на доске.
Первая строка содержит целое число $n$ --- длину перестановки ($3 \le n \le 1000$).
Вторая строка содержит $n$ различных целых чисел, каждое из которых лежит
в диапазоне от 1 до $n$ --- саму перестановку. 

\OutputFile

В первой строке выведите число $k$ --- количество операций, которое необходимо сделать Васе.
В следующей строке выведите $k$ чисел --- саму последовательность операций. Если на очередном
шаге надо поменять местами $i$-й и $i+1$-й элементы перестановки, необходимо вывести число $i$.

Если ответов несколько, вы можете вывести любой. Обратите внимание, что вам не обязательно
минимизировать количество операций. Достаточно, чтобы оно было не больше, чем $n$.
Если решения не существует, выведите число $-1$. 

\Examples

\begin{example}%
\exmp{
5
1 2 3 4 5
}{
3
4 1 2 
}%
\exmp{
5
1 2 3 4 5
}{
2
4 2         
}%
\exmp{
5
2 3 1 5 4
}{
0
}%
\end{example}

\Note

В первом примере, перестановка будет иметь такой вид
$\langle 1 2 3 4 5\rangle \rightarrow \langle 1 2 3 5 4\rangle \rightarrow \langle 2 1 3 5 4\rangle \rightarrow \langle 2 3 1 5 4\rangle$.
Ответ, предложенный во втором примере, тоже корректен.

В третьем примере перестановка уже хаотическая, и ничего менять не надо.

\end{problem}
