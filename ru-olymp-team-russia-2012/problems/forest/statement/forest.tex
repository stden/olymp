\begin{problem}{Лесопосадки}{forest.in}{forest.out}{2 секунды}{256 мегабайт}

%Автор задачи: Демид Кучеренко
%Автор условия: Павел Кротков

Часто при вырубке леса для строительства, например, дорог проводятся компенсационные
лесопосадки~--- равное или большее число деревьев высаживается в другом месте.
Однако посаженный таким образом лес отличается от самостоятельно выросшего~--- 
при компенсационной посадке деревья обычно сажают в узлах регулярной решетки, 
что изменяет влияние деревьев друг на друга.

Павел занимается математическим моделированием роста деревьев, высаженных
в узлах прямоугольной решетки.
Будем считать, что в узлы решетки размером $n\times m$ высажены деревья. 
Два дерева считаются соседними, если узлы, в которых они растут, являются соседними
по вертикали или по горизонтали. У каждого дерева есть своя высота, 
равная некоторому целому числу метров.

Павел считает, что высота деревьев изменяется с годами по следующему закону: 
\begin{itemize}
\item если у дерева есть хотя бы один сосед, высота которого ровно на один 
метр больше высоты самого дерева, то через год высота дерева увеличится 
ровно на один метр;
\item если у дерева нет такого соседа, его высота через год останется прежней.
\end{itemize}

При моделировании роста деревьев по этим правилам рост всех деревьев остановится 
в тот момент, когда соседних деревьев с разницей в росте, равной одному метру, 
не останется. 
Павел хочет выяснить, сколько лет пройдет до этого момента и какая высота будет в 
этот момент у каждого дерева. 

Помогите ему, напишите программу, которая по заданным начальным высотам деревьев
выяснит, через сколько лет рост всех деревьев остановится и какой будет высота каждого
дерева в этот момент.

\InputFile
В первой строке входного файла находятся два целых числа $n$ 
и $m$~--- размеры участка леса ($1 \le n, m \le 100$). 
Следующие $n$ строк содержат по $m$ натуральных чисел, каждое из которых задает
высоту соответствующего дерева. Высота каждого дерева не превышает $100$.

\OutputFile                        
В первой строке выходного файла выведите $t$~--- число лет, которое пройдет 
до того момента, когда все деревья перестанут расти.
После этого выведите $n$ строк по $m$ чисел~--- 
каждое число должно быть равно высоте соответствующего дерева через $t$ лет. 

\Example
\begin{example}%
\exmp{
3 4
1 1 1 2
1 5 5 1
3 1 1 1
}{
9
3 3 3 3
3 5 5 3
3 3 3 3
}%
\end{example}

\end{problem}
