\begin{problem}{Топот котов}{loudcats.in}{loudcats.out}{2 секунды}{256 мегабайт}

%Автор задачи: Максим Буздалов
%Автор условия: Борис Минаев

В одном городе люди постоянно жаловались на то, что им мешают спать. Каждый день у 
соответствующих чиновников собиралась большая куча заявлений о слишком громком 
поведении некоторых людей ночью. С этим необходимо было что-то делать. Тогда на
очередном собрании было решено принять закон, который запрещает издавать громкие
звуки после одиннадцати часов вечера. 

В соответствии с бюрократическими традициями, закон должен содержать расшифровку
понятия <<громкий звук>>. В результате обсуждения, ночью решили запретить, 
например, играть на музыкальных инструментах, передвигать мебель, забивать гвозди.

Когда закон уже собирались принимать, один депутат заметил, что холодильник не 
является мебелью, и его перемещение не попадает под действие закона. Другие депутаты 
также начали придумывать дополнительные запреты, которые исходно не попали в закон. 
В результате были запрещены ночные стоны, скрипы, лай собак и \emph{топот котов}.

За нарушение закона был введен штраф в размере $a$ рублей. 

Узнав о законе, Петя решил выяснить, какой штраф может быть наложен
на жильцов его дома. Дом, в котором живет Петя, имеет $n$ этажей, 
на каждом этаже находится по $m$ квартир. Квартиры в доме пронумерованы от 1 до $nm$. 
Если на некотором не последнем этаже находится квартира номер $x$, то 
непосредственно над ней расположена квартира номер $x + m$. 

Известно, что в $i$-й квартире живет $b_i$ котов.
Петя предположил, что жильцы некоторой квартиры будут жаловаться на соседей
сверху только в том случае, если коты сверху топают \emph{существенно громче}, 
чем их собственные. Проведя эксперименты, Петя решил, что $p$ котов
топают существенно громче, чем $q$ котов, если $p > 2q$.

Выясните, какой суммарный штраф придется заплатить жильцам этого дома,
если все, у кого коты в квартире непосредственно сверху топают существенно
громче, чем их собственные коты, пожалуются на своих соседей сверху
и на тех будет наложен штраф.

\InputFile
Первая строка входного файла содержит три целых числа 
$n$, $m$, $a$~--- количество этажей, количество
квартир на каждом этаже и размер штрафа ($1 \le n \le 20$, 
$1 \le m \le 10$, $1 \le a \le 1000$). 
В следующей строке содержится $nm$ целых чисел $b_1, b_2, \ldots, b_{nm}$, 
где $b_i$~--- количество котов 
в $i$-й квартире ($1 \le b_i \le 30$). 

\OutputFile
Выведите в выходной файл искомый суммарный штраф.

\Example
\begin{example}%
\exmp{
2 3 10
3 5 2 4 10 5
}{
10
}%
\end{example}

\Note
В примере штраф придется заплатить только жильцам 6-й квартиры.

\end{problem}
